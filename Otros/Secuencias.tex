Listado de secuencias mas comunes y como hallarlas.

\begin{center}
\tablefirsthead{}
\tabletail{
\midrule 
\multicolumn{2}{r}{{Continúa en la siguiente columna}} \\}
\tablelasttail{}
{\renewcommand{\arraystretch}{1.4}
\begin{supertabular}{|p{1.8cm}|p{8.6cm}|}

\hline

\multirow{2}{2cm}{Estrellas octangulares}
& 0, 1, 14, 51, 124, 245, 426, 679, 1016, 1449, 1990, 2651, ...
\\ \cline{2-2}
& $f(n) = n*(2*n^{2} - 1)$.
\\ \hline

\multirow{2}{2cm}
{Euler totient}    
& 1, 1, 2, 2, 4, 2, 6, 4, 6, 4, 10, 4, 12, 6,...            
\\ \cline{2-2} 
& $f(n) = $ Cantidad de números naturales $\leq n$ coprimos con n. 
\\ \hline

\multirow{2}{2cm}{Números de Bell} 
& 1, 1, 2, 5, 15, 52, 203, 877, 4140, 21147, 115975, ...
\\ \cline{2-2} 
& Se inicia una matriz triangular con f[0][0] = f[1][0] = 1. La suma de estos dos se guarda en f[1][1] y se traslada a f[2][0]. Ahora se suman f[1][0] con f[2][0] y se guarda en f[2][1]. Luego se suman f[1][1] con f[2][1] y se guarda en f[2][2] trasladandose a f[3][0] y así sucesivamente. Los valores de la primera columna contienen la respuesta.
\\ \hline

\multirow{2}{2cm}
{Números de Catalán} 
& 1, 1, 2, 5, 14, 42, 132, 429, 1430, 4862, 16796, 58786, ...
\\ \cline{2-2}
& $f(n)=\displaystyle\frac{(2n)!}{(n + 1)! n!}$

\\ \hline

\multirow{2}{2cm}{Números de Fermat}
& 3, 5, 17, 257, 65537, 4294967297, 18446744073709551617, ...
\\ \cline{2-2}
& $f(n) = 2^{(\displaystyle2^{\textstyle n})} + 1$
\\ \hline

\multirow{2}{2cm}
{Números de Fibonacci} 
& 0, 1, 1, 2, 3, 5, 8, 13, 21, 34, 55, 89, 144, 233, ...    
\\ \cline{2-2} 
& $f(0) = 0$; $f(1) = 1$; $f(n) = f(n-1) + f(n-2)$ para $n>1$             \\ \hline

\multirow{2}{2cm}
{Números de Lucas} 
& 2, 1, 3, 4, 7, 11, 18, 29, 47, 76, 123, 199, 322, ...    
\\ \cline{2-2} 
& $f(0) = 2$; $f(1) = 1$; $f(n) = f(n-1) + f(n-2)$ para $n>1$            
\\ \hline

\multirow{2}{2cm}{Números de Pell} 
& 0, 1, 2, 5, 12, 29, 70, 169, 408, 985, 2378, 5741, 13860, ...
\\ \cline{2-2} 
& $f(0) = 0; f(1) = 1; f(n) = 2f(n-1) + f(n-2)$ para $n>1$
\\ \hline

\multirow{2}{2cm}
{Números de Tribonacci} 
& 0, 0, 1, 1, 2, 4, 7, 13, 24, 44, 81, 149, 274, 504, ...    
\\ \cline{2-2} 
& $f(0)=f(1)=0; f(2)=1; f(n) = f(n-1) + f(n-2) + f(n-3)$ para $n>2$
\\ \hline

\multirow{2}{2cm}{Números factoriales}
& 1, 1, 2, 6, 24, 120, 720, 5040, 40320, 362880, ...
\\ \cline{2-2}
&$ f(0) = 1; f(n) = \displaystyle\prod_{\textstyle k=1}^{\textstyle n}k$ para $n>0$.

\\ \hline

\multirow{2}{2cm}{Números piramidales cuadrados}
& 0, 1, 5, 14, 30, 55, 91, 140, 204, 285, 385, 506, 650, ...
\\ \cline{2-2}
& $f(n) = \displaystyle\frac{n*(n+1)*(2*n+1)}{6}$

\\ \hline

\multirow{2}{2cm}{Números primos de Mersenne}
& 3, 7, 31, 127, 8191, 131071, 524287, 2147483647, ...
\\ \cline{2-2}
& $f(n) = 2^{p(n)} - 1$ donde $p$ representa valores primos iniciando en $p(0)=2$.
\\ \hline


\multirow{2}{2cm}{Números tetraedrales}
& 1, 4, 10, 20, 35, 56, 84, 120, 165, 220, 286, 364, 455,  ...
\\ \cline{2-2}
& $f(n) = \displaystyle\frac{n*(n+1)*(n+2)}{6}$

\\ \hline

\multirow{2}{2cm}{Números triangulares}
& 0, 1, 3, 6, 10, 15, 21, 28, 36, 45, 55, 66, 78, 91, 105, ...
\\ \cline{2-2}
& $f(n) = \displaystyle\frac{n(n+1)}{2}$

\\ \hline

\multirow{2}{2cm}{OEIS A000127}
& 1, 2, 4, 8, 16, 31, 57, 99, 163, 256, 386, 562, ...
\\ \cline{2-2}
& $f(n) = \displaystyle\frac{(n^{4}-6n^{3}+23n^{2}-18{n}+24)}{24}$.

\\ \hline

\multirow{2}{2cm}{Secuencia de Narayana}
& 1, 1, 1, 2, 3, 4, 6, 9, 13, 19, 28, 41, 60, 88, 129, ...
\\ \cline{2-2}
& $f(0) = f(1) = f(2) = 1; f(n) = f(n-1) + f(n-3)$ para todo $n>2$.
\\ \hline

\multirow{2}{2cm}
{Secuencia de Silvestre} 
& 2, 3, 7, 43, 1807, 3263443, 10650056950807, ...    
\\ \cline{2-2} 
& $f(0) = 2; f(n+1) = f(n)^2 - f(n) + 1$
\\ \hline

\multirow{2}{2cm}{Secuencia de vendedor perezoso} 
& 1, 2, 4, 7, 11, 16, 22, 29, 37, 46, 56, 67, 79, 92, 106, ...
\\ \cline{2-2} 
& Equivale al triangular(n) + 1. Máxima número de piezas que se pueden formar al hacer n cortes a un disco. 

$f(n) = \displaystyle\frac{n(n+1)}{2} + 1$

\\ \hline

\multirow{2}{2cm}{Suma de los divisores de un número}
& 1, 3, 4, 7, 6, 12, 8, 15, 13, 18, 12, 28, 14, 24, ...
\\ \cline{2-2}
&Para todo $n>1$ cuya descomposición en factores primos es $n=\displaystyle p_{1}^{\textstyle a_{1}}\displaystyle p_{2}^{\textstyle a_{2}}...\displaystyle p_{k}^{\textstyle a_{k}}$ se tiene que:

$f(n) = \displaystyle\frac{p_{1}^{a_{1} + 1} - 1}{p_{1} - 1} * \frac{p_{2}^{a_{2} + 1} - 1}{p_{2} - 1} * ... * \frac{p_{k}^{a_{k} + 1} - 1}{p_{k} - 1}$ 

\\ \hline
\end{supertabular}
}
\end{center}
